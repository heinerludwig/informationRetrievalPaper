\documentclass{sigchi-ext}
% Please be sure that you have the dependencies (i.e., additional
% LaTeX packages) to compile this example.
\usepackage[T1]{fontenc}
\usepackage{textcomp}
\usepackage[scaled=.92]{helvet} % for proper fonts
\usepackage{graphicx} % for EPS use the graphics package instead
\usepackage{balance}  % for useful for balancing the last columns
\usepackage{booktabs} % for pretty table rules
\usepackage{ccicons}  % for Creative Commons citation icons
\usepackage{ragged2e} % for tighter hyphenation

% Some optional stuff you might like/need.
% \usepackage{marginnote} 
% \usepackage[shortlabels]{enumitem}
% \usepackage{paralist}
% \usepackage[utf8]{inputenc} % for a UTF8 editor only

%% EXAMPLE BEGIN -- HOW TO OVERRIDE THE DEFAULT COPYRIGHT STRIP --
% \copyrightinfo{Permission to make digital or hard copies of all or
% part of this work for personal or classroom use is granted without
% fee provided that copies are not made or distributed for profit or
% commercial advantage and that copies bear this notice and the full
% citation on the first page. Copyrights for components of this work
% owned by others than ACM must be honored. Abstracting with credit is
% permitted. To copy otherwise, or republish, to post on servers or to
% redistribute to lists, requires prior specific permission and/or a
% fee. Request permissions from permissions@acm.org.\\
% {\emph{CHI'14}}, April 26--May 1, 2014, Toronto, Canada. \\
% Copyright \copyright~2014 ACM ISBN/14/04...\$15.00. \\
% DOI string from ACM form confirmation}
%% EXAMPLE END

% Paper metadata (use plain text, for PDF inclusion and later
% re-using, if desired).  Use \emtpyauthor when submitting for review
% so you remain anonymous.
\def\plaintitle{SIGCHI Extended Abstracts Sample File: Note Initial
  Caps} \def\plainauthor{First Author, Second Author, Third Author,
  Fourth Author, Fifth Author, Sixth Author}
\def\emptyauthor{}
\def\plainkeywords{Authors' choice; of terms; separated; by
  semicolons; include commas, within terms only; required.}
\def\plaingeneralterms{Documentation, Standardization}

\title{SIGCHI Extended Abstracts Sample File: \underline{N}ote
  \underline{I}nitial \underline{C}aps}

\numberofauthors{6}
% Notice how author names are alternately typesetted to appear ordered
% in 2-column format; i.e., the first 4 autors on the first column and
% the other 4 auhors on the second column. Actually, it's up to you to
% strictly adhere to this author notation.
\author{%
  \alignauthor{%
    \textbf{Alexandra}\\ 
    \affaddr{Interactive Media Lab} \\
    \affaddr{Technische Universit�t Dresden} \\
    \affaddr{Dresden, Germany} \\
    \email{author1@anotherco.edu} } \vfil \alignauthor{%
    \textbf{Maxime}\\
    \affaddr{VP, Authoring}\\
    \affaddr{Authorship Holdings, Ltd.}\\
    \affaddr{Awdur SA22 8PP, UK}\\
    \email{author2@author.ac.uk} } \vfil \alignauthor{%
    \textbf{Heiner}\\
   \affaddr{Interactive Media Lab} \\
    \affaddr{Technische Universit�t Dresden} \\
    \affaddr{Dresden, Germany} \\
    \email{heiner.ludwig@tu-dresden.de} }}

% Make sure hyperref comes last of your loaded packages, to give it a
% fighting chance of not being over-written, since its job is to
% redefine many LaTeX commands.
\definecolor{linkColor}{RGB}{6,125,233}
\hypersetup{%
  pdftitle={\plaintitle},
%  pdfauthor={\plainauthor},
  pdfauthor={\emptyauthor},
  pdfkeywords={\plainkeywords},
  bookmarksnumbered,
  pdfstartview={FitH},
  colorlinks,
  citecolor=black,
  filecolor=black,
  linkcolor=black,
  urlcolor=linkColor,
  breaklinks=true,
}

% \reversemarginpar%

\begin{document}

%% For the camera ready, use the commands provided by the ACM in the Permission Release Form.
%\CopyrightYear{2016}
%\setcopyright{rightsretained}
%\conferenceinfo{WOODSTOCK}{'97 El Paso, Texas USA}
%\isbn{0-12345-67-8/90/01}
%\doi{http://dx.doi.org/10.1145/2858036.2858119}
%% Then override the default copyright message with the \acmcopyright command.
%\copyrightinfo{\acmcopyright}

\maketitle

% Uncomment to disable hyphenation (not recommended)
% https://twitter.com/anjirokhan/status/546046683331973120
\RaggedRight{} 

% Do not change the page size or page settings.
\begin{abstract}
  UPDATED---\today. This sample paper describes the formatting
  requirements for SIGCHI Extended Abstract Format, and this sample
  file offers recommendations on writing for the worldwide SIGCHI
  readership. Please review this document even if you have submitted
  to SIGCHI conferences before, as some format details have changed
  relative to previous years. Abstracts should be about 150
  words. Required.
\end{abstract}

\keywords{\plainkeywords}

\category{H.5.m}{Information interfaces and presentation (e.g.,
  HCI)}{Miscellaneous}\category{See}{\url{http://acm.org/about/class/1998/}}{for
  full list of ACM classifiers. This section is required.}

\section{Einf�hrung}
This format is to be used for submissions that are published in the
conference publications. We wish to give this volume a consistent,
high-quality appearance. We therefore ask that authors follow some
simple guidelines. In essence, you should format your paper exactly
like this document. The easiest way to do this is to replace the
content with your own material.

\section{Related Work}
\subsubsection{Gesten}
Gestik als Eingabemedium zur Erstellung von Phantombildern ist g�nzlich neu, der Aufbau einer geeigneten Gesten-Sprache demnach unumg�nglich. 

Das Buch Emotionales Interaktionsdesign besch�ftigt sich
 u.a. mit der Semantik von Gestensprachen und zeigt Notwendigkeiten wie Abbruch-Gesten oder die allgemeine Robustheit von Gesten auf [1]. Kaisa V��n�nen und Klaus B�hm weisen auf Schwierigkeiten wie Erm�dung der Gliedma{\ss}en und eingeschr�nkte Exaktheit w�hrend der Gesten-Nutzung hin, betonen jedoch gleichzeitig die vereinfachte Navigation und Manipulation f�r verschiedenste Nutzergruppen in virtuellen 3D-R�umen[X].
In [] ist laut einer Nutzerstudie die Metaphern-Nutzung bei Gesten weniger effektiv als die Nutzung von Freihand-Gesten. 
Gem�� [7,8] ist allgemein die Nutzung von formalisierten Gesten (?semaphoric gestures?) nicht sinnvoll, da sie w�hrend der Interaktion unnat�rlich wirkt.
[S.54] besagt jedoch, dass beim Gro�teil der gestenbasierten Eingabesysteme sowohl formalisierte, als auch Freihand-Gesten genutzt werden, bspw. bei Interaktionen �ber gr��ere Distanzen.
Nicht zuletzt haben Hollywood-Filme wie Iron Man oder Minority Report zur Inspiration f�r die User Experience und bei der Erstellung der Nutzungsoberfl�che gedient[].

Zur Kombination von Virtueller Realit�t mit Gestenmanipulation haben die Firmen \textit{eyeSight\footnote{\url{eyesight-tech.com/vr-ar.html}}} und 
\textit{pebbles\footnote{\url{www3.oculus.com/en-us/blog/pebbles-interfaces-joins-oculus/}}} unabh�ngig voneinander eine VR-Brille mit mehreren Kameras kombiniert um die Interaktion mit dem virtuellen Raum f�r den Nutzer m�glich zum machen.

\begin{marginfigure}[-35pc]
  \begin{minipage}{\marginparwidth}
    \centering
    \includegraphics[width=0.9\marginparwidth]{figures/cats}
    \caption{In this image, the cats are tessellated within a square
      frame. Images should also have captions and be within the
      boundaries of the sidebar on page~\pageref{sec:sidebar}. Photo:
      \cczero~jofish on Flickr.}~\label{fig:marginfig}
  \end{minipage}
\end{marginfigure}


\includegraphics[width=1\marginparwidth]{figures/system_overview}

\section{Konzept}

Die Idee zur digitalen Erstellung von Phantombildern ist ein digitales System, welches den Nutzer in eine virtuelle Realit�t versetzt, die mittels Gesten manipulierbar ist. Durch die Interaktion mit Suchanfragen- und Ergebnisbest�ckten Objekten steuert der Nutzer die Erstellung des Phantombildes, indem er diese Objekte ausw�hlt, verschiebt, l�scht oder kombiniert. 

Der Nutzer navigiert das System von einem Punkt aus. Um einzelne Objekte genauer zu betrachten, bewegt er sich also nicht k�rperlich, sondern ver�ndert die Position des virtuellen Raums mittels Translations- und Skalierungsgesten.

Die Objekte sind im 180-Winkel "schwebend" um den Nutzer angeordnet(Bild). Durch sowohl freie, als auch objektorientierte Gesten, kann er, mittels ein- oder zweih�ndiger Gestikulation, Suchanfragen, Ergebnisse oder Ergebnisvorschl�ge speichern, l�schen, zum Ergebnis-pool hinzuf�gen, neu kombinieren oder r�ckg�ngig machen.
Das System wird immer nur von einem Nutzer gleichzeitig benutzt,
weitere Personen k�nnen w�hrend des gesamten Findungsprozesses
ausschlie�lich als Zuschauer involviert werden. So ist eine
Moderatorenrolle eines Experten denkbar, der Hilfestellung gibt.

Aus der Sicht der Nutzbarkeit bietet das System folgende Vorteile:
\begin{itemize}\compresslist%
\item Effizientes und selbstst�ndiges Arbeiten durch intuitive Gesten und Hilfestellung durch das System.
\item Sehr direkte Interaktion vom Nutzer mit der Benutzungsoberfl�che.
\item Effektive Ergebnisfindung durch st�ndiges Feedback und Anpassung vom System.
\item Nutzerbefriedigung durch immer sichtbaren Fortschritt.
\item Einsatz ist ortsunabh�ngig und verlangt nur kleine R�umlichkeiten
\end{itemize}

\section{Text Formatting}


The \LaTeX\ template facilitates text formatting for normal (for body
text); heading 1, heading 2, heading 3; bullet list; numbered list;
caption; annotation (for notes in the narrow left margin); and
references (for bibliographic entries). Additionally, here is an
example of footnoted\footnote{Use footnotes sparingly, if at all.}
text. As stated in the footnote, footnotes should rarely be used.

\begin{itemize}\compresslist%
\item Write in a straightforward style. Use simple sentence
  structure. Try to avoid long sentences and complex sentence
  structures. Use semicolons carefully.
\item Use common and basic vocabulary (e.g., use the word ``unusual''
  rather than the word ``arcane'').
\item Briefly define or explain all technical terms. The terminology
  common to your practice/discipline may be different in other design
  practices/disciplines.
\item Spell out all acronyms the first time they are used in your
  text. For example, ``World Wide Web (WWW)''.
\item Explain local references (e.g., not everyone knows all city
  names in a particular country).
\item Explain ``insider'' comments. Ensure that your whole audience
  understands any reference whose meaning you do not describe (e.g.,
  do not assume that everyone has used a Macintosh or a particular
  application).
\item Explain colloquial language and puns. Understanding phrases like
  ``red herring'' requires a cultural knowledge of English. Humor and
  irony are difficult to translate.
\item Use unambiguous forms for culturally localized concepts, such as
  times, dates, currencies, and numbers (e.g., ``1-5- 97'' or
  ``5/1/97'' may mean 5 January or 1 May, and ``seven o'clock'' may
  mean 7:00 am or 19:00). For currencies, indicate equivalences:
  ``Participants were paid {\fontfamily{txr}\selectfont \textwon}
  25,000, or roughly US \$22.''
\item Be careful with the use of gender-specific pronouns (he, she)
  and other gender-specific words (chairman, manpower,
  man-months). Use inclusive language (e.g., she or he, they, chair,
  staff, staff-hours, person-years) that is gender-neutral. If
  necessary, you may be able to use ``he'' and ``she'' in alternating
  sentences, so that the two genders occur equally
  often~\cite{Schwartz:1995:GBF}.
\item If possible, use the full (extended) alphabetic character set
  for names of persons, institutions, and places (e.g.,
  Gr{\o}nb{\ae}k, Lafreni\'ere, S\'anchez, Nguy{\~{\^{e}}}n,
  Universit{\"a}t, Wei{\ss}enbach, Z{\"u}llighoven, \r{A}rhus, etc.).
  These characters are already included in most versions and variants
  of Times, Helvetica, and Arial fonts.
\end{itemize}

% \begin{figure}
%   \includegraphics[width=.9\columnwidth]{figures/ea-figure2}
%   \caption{If your figure has a light background, you can set its
%     outline to light gray, like this, to make a box around
%     it.}\label{fig:bats}
% \end{figure}

\begin{marginfigure}[-35pc]
  \begin{minipage}{\marginparwidth}
    \centering
    \includegraphics[width=0.9\marginparwidth]{figures/cats}
    \caption{In this image, the cats are tessellated within a square
      frame. Images should also have captions and be within the
      boundaries of the sidebar on page~\pageref{sec:sidebar}. Photo:
      \cczero~jofish on Flickr.}~\label{fig:marginfig}
  \end{minipage}
\end{marginfigure}

\section{Figures}
The examples on this and following pages should help you get a feel
for how screen-shots and other figures should be placed in the
template. Your document may use color figures (see
Figures~\ref{fig:sample}), which are included in the page limit; the
figures must be usable when printed in black and white. You can use
the \texttt{\marginpar} command to insert figures in the (left) margin
of the document (see Figure~\ref{fig:marginfig}). Finally, be sure to
make images large enough so the important details are legible and
clear (see Figure~\ref{fig:cats}).

\section{Tables}
You man use tables inline with the text (see Table~\ref{tab:table1})
or within the margin as shown in Table~\ref{tab:table2}. Try to
minimize the use of lines (especially vertical lines). \LaTeX\ will
set the table font and captions sizes correctly; the latter must
remain unchanged.

\section{Accessibility}
The Executive Council of SIGCHI has committed to making SIGCHI
conferences more inclusive for researchers, practitioners, and
educators with disabilities. As a part of this goal, the all authors
are asked to work on improving the accessibility of their
submissions. Specifically, we encourage authors to carry out the
following five steps:
\begin{itemize}\compresslist%
\item Add alternative text to all figures
\item Mark table headings
\item Generate a tagged PDF
\item Verify the default language
\item Set the tab order to ``Use Document Structure''
\end{itemize}

For links to instructions and resources, please see:
\url{http://chi2016.acm.org/accessibility}

Unfortunately good tools do not yet exist to create tagged PDF files
from Latex. \LaTeX\ users will need to carry out all of the above
steps in the PDF directly using Adobe Acrobat, after the PDF has been
generated.

For more information and links to instructions and resources, please
see:
\url{http://chi2016.acm.org/accessibility}.

\begin{figure*}
  \centering
  \includegraphics[width=1.3\columnwidth]{figures/map}
  \caption{In this image, the map maximizes use of space. You can make
    figures as wide as you need, up to a maximum of the full width of
    both columns. Note that \LaTeX\ tends to render large figures on a
    dedicated page. Image: \ccbynd~ayman on Flickr.}~\label{fig:cats}
\end{figure*}

\section{Producing and Testing PDF Files}
We recommend that you produce a PDF version of your submission well
before the final deadline. Your PDF file must be ACM DL Compliant and
meet stated requirements,
\url{http://www.sheridanprinting.com/sigchi/ACM-SIG-distilling-settings.htm}.

\marginpar{\vspace{-23pc}So long as you don't type outside the right
  margin or bleed into the gutter, it's okay to put annotations over
  here on the left, too; this annotation is near Hawaii. You'll have
  to manually align the margin paragraphs to your \LaTeX\ floats using
  the \texttt{{\textbackslash}vspace{}} command.}

\begin{margintable}[1pc]
  \begin{minipage}{\marginparwidth}
    \centering
    \begin{tabular}{r r l}
      & {\small \textbf{First}}
      & {\small \textbf{Location}} \\
      \toprule
      Child & 22.5 & Melbourne \\
      Adult & 22.0 & Bogot\'a \\
      \midrule
      Gene & 22.0 & Palo Alto \\
      John & 34.5 & Minneapolis \\
      \bottomrule
    \end{tabular}
    \caption{A simple narrow table in the left margin
      space.}~\label{tab:table2}
  \end{minipage}
\end{margintable}
Test your PDF file by viewing or printing it with the same software we
will use when we receive it, Adobe Acrobat Reader Version 10. This is
widely available at no cost. Note that most
reviewers will use a North American/European version of Acrobat
reader, so please check your PDF accordingly.

\section{Acknowledgements}
We thank all the volunteers, publications support, staff, and authors
who wrote and provided helpful comments on previous versions of this
document. As well authors 1, 2, and 3 gratefully acknowledge the grant
from NSF (\#1234--2222--ABC). Author 4 for example may want to
acknowledge a supervisor/manager from their original employer. This
whole paragraph is just for example. Some of the references cited in
this paper are included for illustrative purposes only.

\section{References Format}
Your references should be published materials accessible to the
public. Internal technical reports may be cited only if they are
easily accessible and may be obtained by any reader for a nominal
fee. Proprietary information may not be cited. Private communications
should be acknowledged in the main text, not referenced (e.g.,
[Golovchinsky, personal communication]). References must be the same
font size as other body text. References should be in alphabetical
order by last name of first author. Use a numbered list of references
at the end of the article, ordered alphabetically by last name of
first author, and referenced by numbers in brackets. For papers from
conference proceedings, include the title of the paper and the name of
the conference. Do not include the location of the conference or the
exact date; do include the page numbers if available. 

References should be in ACM citation format.  This
includes citations to Internet
resources~\cite{CHINOSAUR:venue,cavender:writing,psy:gangnam}
according to ACM format, although it is often appropriate to include
URLs directly in the text, as above. Example reference formatting for
individual journal articles~\cite{ethics}, articles in conference
proceedings~\cite{Klemmer:2002:WSC:503376.503378},
books~\cite{Schwartz:1995:GBF}, theses~\cite{sutherland:sketchpad},
book chapters~\cite{winner:politics}, an entire journal
issue~\cite{kaye:puc},
websites~\cite{acm_categories,cavender:writing},
tweets~\cite{CHINOSAUR:venue}, patents~\cite{heilig:sensorama}, 
games~\cite{supermetroid:snes}, and
online videos~\cite{psy:gangnam} is given here.  See the examples of
citations at the end of this document and in the accompanying
\texttt{BibTeX} document. This formatting is a edited version of the
format automatically generated by the ACM Digital Library
(\url{http://dl.acm.org}) as ``ACM Ref''. DOI and/or URL links are
optional but encouraged as are full first names. Note that the
Hyperlink style used throughout this document uses blue links;
however, URLs in the references section may optionally appear in
black.

\balance{} 

\bibliographystyle{SIGCHI-Reference-Format}
\bibliography{sample}

\end{document}

%%% Local Variables:
%%% mode: latex
%%% TeX-master: t
%%% End:
